\documentclass{ees}

\shorttitle{Missa Vix orimur}

\begin{document}

\eesTitlePage

\eesCriticalReport{
  – & –  & –      & \B1 contains chorus and org parts throughout, while
                    instrumental parts are often absent. These presumably
                    missing parts have been reconstructed by the editor
                    (for details, see remarks on individual movements below).
                    An oboe only appears in the \textit{Gloria}, where it plays
                    unison with S; thus, an analogous ob part has also been
                    added to the remaining movements. \\
  \midrule
  1 & –  & –      & Parts for vl 1 (unison with S), vl 2 (unison with A),
                    and vla (unison with~T) have been added by the editor. \\
    & – & chorus  & While “eleison” is always hyphenated “e-lei-son” in \B1,
                    this edition rather uses “e-le-i-son” whenever the rhythm
                    implicates such a hyphenation. \\
  \midrule
  2 & –  & –      & vla is indicated by “col Tenore” in bars 1–43 and 45–53.
                    Here, it pauses during T solo sections. \\
    & 13 & vl     & 2nd \quarterNote\ in \B1: e″4 \\
    & 54 & clno 2 & last \quarterNote\ missing in \B1 \\
    & 55 & clno 1 & 1st \quarterNote\ in \B1: e″4 \\
  \midrule
  3 & –  & –      & clno 3 and cor 1/2 only appear in the \textit{Credo}. \\
    & 14 & S      & 1st \quarterNote\ in \B1: d″4 \\
    & 24 & vl, A  & bar in \B1: e′2–\crotchetRest \\
    & 48 & vl 1   & 2nd \quarterNote\ missing in \B1 \\
    & 53 & org    & 1st \wholeNote\ in \B1: B1 \\
    & 72 & org    & lower voice missing in \B1 \\
  \midrule
  4 & –  & –      & Parts for clno 1/2 and timp (editor’s suggestion),
                    as well as vl 1 (unison with A, one octave higher),
                    vl 2 (unison with S), and vla (unison with T)
                    have been added by the editor. \\
    & 29 & A      & 2nd \halfNote\ in \B1: \sharp f′2 \\
  \midrule
  5 & –  & –      & Parts for clno 1/2 and timp (editor’s suggestion),
                    as well as vl 1 (unison with A, one octave higher),
                    vl 2 (unison with S), and vla (unison with T)
                    have been added by the editor. \\
    & 32 & T      & 2nd/3rd \halfNote\ in \B1: e′2–\sharp f′2 \\
    & 34 & A      & 2nd \halfNote\ in \B1: g′2 \\
}

\eesToc{}

\eesScore

\end{document}
